\chapter{Background}
En este capítulo pasamos a detallar todos aquellos elementos necesarios para comprender este proyecto.
En primer lugar introduciremos la  competición The General Video Game AI Competition (GVG-AI) \cite{gvgai}. Donde se definen las normas que rigen nuestro agente. A continuación detallamos conceptos calve de planificación explicando con más detalle el algoritmo escogido para nuestra implementación

\section{GVG-AI}
The GVG-AI Competition explores the problem of creating controllers for general video game playing.
This controllers must be able to play different games without previous knowledge about them.
The framework of this competition show us information about the games in an object-oriented manner, and employs a Video Game Description Language (VGDL) \ref{vgdl-section} to define games in a more general way.

In this enviroment, the agent receives
calls at every game step and must return a discrete
action to apply in no more than 40ms. In this period we need explore the different paths or combinations of actions to obtain the best rewards and win the games. That limit don't allow as to explore all the possibles combinations and for this reason we need to use an strategy that could obtain a good option and avoid "the Game Over"

\subsection{Filosofy}

????? ????????????? ????????????? ????????????? ????????????? ????????????? 

\subsection{Simulator}

????? ????????????? ????????????? ????????????? ????????????? ????????????? 

\subsection{VGDL} \label{vgdl-section}

????? ????????????? ????????????? ????????????? ????????????? ????????????? 


\section{IW}


????? ????????????? ????????????? ????????????? ????????????? ????????????? 
\subsection{Planing concepts}

????? ????????????? ????????????? ????????????? ????????????? ????????????? 

\subsection{Algorithm IW}
????? ????????????? ????????????? ????????????? ????????????? ????????????? 